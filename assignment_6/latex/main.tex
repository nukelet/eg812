\documentclass{article}

\usepackage[portuguese]{babel}

\usepackage{amsmath, amssymb}
\usepackage{graphicx}
\usepackage[colorlinks=true, allcolors=blue]{hyperref}

\usepackage[section]{placeins}

\title{Lista 06}
\author{Vinícius de Oliveira Peixoto Rodrigues (245294)}
\date{Setembro de 2022}

\begin{document}
\maketitle

\section*{Questão 1}

A expansão ocorre porque, no RSA, o texto cifrado é calculado a partir da chave pública $(e, n)$ como:

\begin{equation*}
    c = m^e \mod n
\end{equation*}

Normalmente, o valor de $e$ é um número primo relativamente grande (costuma ser $2^{16} + 1$); isso significa que valores de $m$ pequenos (por exemplo, de 1 byte) têm chance de resultar em valores enormes de $c$ (possivelmente da ordem de $n$, que costuma ter centenas de bytes).

Essa expansão pode ser mitigada maximizando o tamanho de "bloco" do plaintext, isto é, cifrando chunks de tamanho comparável ao do valor $n$ (por exemplo, cifrando blocos de 128 bytes para $n$ com 1024 bits).

Para evitar que o arquivo decifrado aumente de tamanho, basta eliminar o padding gerado pela decifração (por exemplo, um caractere que era originalmente de 1 byte no RSA-1024, ao ser decifrado, vai ser um inteiro de 128 bytes, mas basta descartar o "excesso").

\subsection*{Questão 2}

Observando que $35 = 7 \cdot 5 \Rightarrow \phi(35) = (5-1)(7-1) = 24$, temos pelo teorema de Euler que se $\gcd(a, 35) = 1$:

\begin{equation*}
    a^{24} \equiv 1 \mod 35
\end{equation*}

Como $123 = 3 \cdot 41 \Rightarrow \gcd(123, 35) = 1$ e $145 = 6 \cdot 24 + 1$:

\begin{equation*}
    123^{145} \equiv 
    \underbrace{\left(123^{24}\right)^6}_{\equiv 1 \mod 35} \cdot 123 \mod 35 \equiv 123^1 \mod 35 \equiv 18 \mod 35
\end{equation*}

\subsection*{Questão 3}

O valor 2 não seria uma escolha válida porque, por definição, para garantir a unicidade do inverso multiplicativo da chave pública ($ed \equiv \mod \phi(n)$) é condição necessária que $\gcd(e, \phi(n)) = 1$ (senão a chave pública $e$ não teria inverso único). Como $\phi(n) = (p-1)(q-1)$ é necessariamente par (visto que $p$ e $q$ são primos ímpares), a condição de unicidade do inverso não é satisfeita, de modo que a cifração deixa de ser injetiva (e por tanto se torna não-invertível).

\subsection*{Questão 4}

\subsubsection*{1.}
A única forma de determinar com certeza absoluta é por meio de uma checagem direta, testando os divisores até $\sqrt{n}$. É comum que seja usado um algoritmo como o crivo de Eratóstenes para gerar uma "cache" de primos conhecidos até um certo valor.

\subsubsection*{2. e 3.}
A checagem direta tem complexidade $O(\sqrt{n})$, enquanto o crivo de Eratóstenes tem complexidade temporal (em pior caso) de $O(n \log\log n)$ e espacial de $O(n)$. Esse resultaos tornam o método inviável para números muito grandes (como os usados no RSA).

\subsection*{Questão 5}

Não, pela mesma razão mencionada na questão 3: um número de chave pública par faria com que $\gcd(e, \phi(n)) \neq 1$, de modo que a função $f: m \rightarrow m^e \mod n$ deixaria de ser invertível.

\subsection*{Questão 6}

A chance de o número ser primo é dada por $1-0.25^{n}$, para $n$ testes que resultam em "inconclusivo". Dessa forma, queremos que $1-0.25^n = 0.99999 \Rightarrow n = log_{0.25}(1-0.99999) \Rightarrow n \approx 8.3$, de modo que bastam \boxed{\text{9 testes}}.

\subsection*{Questão 7}

Há $2^{1024}$ números de 1024 bits, indo de $2^{1023}$ a $2^{1024} - 1$; a distância "média" esperada entre os primos nesse intervalo seria $\left(\ln(2^{1023}) + \ln(2^{1024})\right)/2 \approx 709.5$.

Dessa forma, é razoável esperar encontrar um número primo após testar em torno de 710 números consecutivos.

\subsection*{Questão 8}

Pelo pequeno teorema de Fermat:

\begin{equation*}
    a^p \equiv a \mod p \Rightarrow a^{p-2} \equiv a^{-1} \mod p
\end{equation*}

De modo que é possível encontrar o inverso multiplicativo calculando $a^{p-2} \mod p$. Para $a = 4$, $p = 11$, temos que
\begin{equation*}
    4^{-1} \equiv 4^9 \mod 11 \equiv 262144 \mod 11 = \boxed{3}
\end{equation*}

É fácil verificar observando que $4 \cdot 3 = 12 \equiv 1 \mod 11$.

\subsection*{Questão 9}

O programa em Python abaixo implementa o square-and-multiply:

\begin{verbatim}
    from math import log, ceil

    def sq_mult(m, e, n):
        s = ceil(log(e, 2))
        t = 1
        for i in reversed(range(0, s)):
            t *= t
            t %= n
            print(f"step {i}: squaring")
            if e & (1 << i):
                t *= m
                t %= n
                print(f"step {i}: multiplying")
        return t
    
    def main():
        m = 42
        print("m = {m}, key = (31, 35)")
        sq_mult(m, 31, 35)
        print("-------------")
        print("m = {m}, key = (33, 35)")
        sq_mult(m, 33, 35)
    
    if __name__ == "__main__":
        main()
     
\end{verbatim}

No primeiro caso, foram realizadas 10 operações (5 squares, 5 multiplies); no segundo, apenas 8 (6 squares, 2 multiplies). Isso acontece porque há tantas operações de multiply quanto há bits 1 na chave pública $e$; como \texttt{31 = 0b11111} e \texttt{33 = 0b100001}, o volume de cálculos é menor (comparativamente ao tamanho de bits da chave pública) para a segunda.

\end{document}
